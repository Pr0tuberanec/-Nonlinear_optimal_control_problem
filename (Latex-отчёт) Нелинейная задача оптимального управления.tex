\documentclass[a4paper,12pt]{article}
\usepackage[utf8]{inputenc}	
\usepackage[T1, T2A]{fontenc}	
\usepackage{indentfirst}	
\usepackage[english, russian]{babel}	
\usepackage{indentfirst}
\usepackage{a4wide}	
\usepackage{amsmath}
\usepackage{amsthm} 
\usepackage{amssymb}
\usepackage{float}
\usepackage{arcs}
\usepackage[top=2cm, bottom=2cm, left=2.5cm, right=1cm]{geometry}
\usepackage{amsfonts}
\usepackage{graphicx}
\usepackage{enumitem}
\usepackage[unicode, pdftex]{hyperref}
%\usepackage[unicode, pdftex]{hyperref}	
\usepackage{setspace}
%\usepackage{vmargin}
\sloppy
\usepackage{indentfirst} % Красная строка
\uchyph=0 % Запрет переноса слова с прописной буквы

\linespread{1.5}
\newtheorem{definition}{Определение}
\newtheorem{thm}{Теорема}
\newcommand{\hm}[1]{#1\nobreak\discretionary{}{\hbox{\ensuremath{#1}}}{}}

\begin{document}
\newcommand{\sgn}{\mathrm{sgn}}

\begin{titlepage}
\begin{center}
\includegraphics[width=8cm, height=4cm]{msu.eps}
\end{center}

\begin{center}
Московский государственный университет имени М.В. Ломоносова\\
\vspace{0.2cm}
Факультет вычислительной математики и кибернетики\\
\vspace{0.2cm}
Кафедра cистемного анализа

\vspace{3.75cm}
{\LARGE Отчёт по практикуму}\\
\vspace{1cm}
{\Huge\bfseries <<Нелинейная задача оптимального управления>>}
\end{center}

\vspace{1.75cm}
\begin{flushright}
\large
\textit{Студент 315 группы}\\
Н.~Ю.~Заварзин\\
\vspace{5mm}
\textit{Руководитель практикума}\\
к.ф.-м.н., доцент П.~A.~Точилин\\
\end{flushright}
\vspace{4cm}

\begin{center}
Москва, 2023
\end{center}
\end{titlepage} 


\newpage

\tableofcontents

\newpage
\section{Постановка задачи}
Движение ракеты в вертикальной плоскости над поверхностью земли описывается дифференциальными уравнениями:
\hypertarget{p1}{}
\begin{equation} \label{eq1}
	\begin{cases} 
		\dot{m}(t) v(t) + m(t)\dot{v}(t) = -gm(t) + lu(t), \\
		\dot{m}(t) = -u(t).
	\end{cases}
\end{equation}

Здесь $v(t) \in \mathbb{R}$~--- скорость ракеты, $m(t)$~--- её переменная масса, $g > 0$~--- гравитационная постоянная, $l > 0$~--- коэффициент, определяющий силу, действующую на ракету со стороны сгорающего топлива, $u(t) \in [0, u_{max}]$~--- скорость подачи топлива ($u_{max} > 0$). Кроме того, известна масса корпуса ракеты без топлива $M > 0$. \\

\textbf{Задача 1.} \textit{Задан начальный момент времени $t_0 = 0$, начальная скорость $v(0) = 0$, а также начальная масса ракеты с топливом $m(0) = m_0 > M$. Необходимо за счёт выбора программного управления $u(t)$ перевести ракету на максимально возможную высоту в заданный момент времени $T > 0$. Кроме того, необходимо, чтобы $v(T) \in [-\varepsilon, \varepsilon]$.}\\

\textbf{Задача 2.} \textit{Задан начальный момент времени $t_0 = 0$, начальная скорость $v(0) = 0$, а также начальная масса ракеты с топливом $m(0) = m_0 > M$. Необходимо за счёт выбора программного управления $u(t)$ перевести ракету на заданную высоту $H > 0$ в заданный момент времени $T > 0$, так, чтобы при этом минимизировать функционал 
\[ \mathcal{J} = \int\limits_{0}^{T}{u^4(t)} \, \mathrm{d}t.\]}
\newpage

\section{Аналитическое решение первой задачи}
Исходная система \eqref{eq1} для первой задачи при $t \in [0, T]$ примет вид
\hypertarget{p2}{}
\begin{equation} \label{eq2}
	\begin{cases} 
		\dot{m}(t) v(t) + m(t)\dot{v}(t) = -gm(t) + lu(t), \\
		\dot{m}(t) = -u(t), \\
		m(0) = m_0, \\
		v(0) = 0, \\
		v(T) \in [-\varepsilon, \varepsilon].
		\end{cases}
\end{equation}

Вводя новые обозначения $x_1(t)$, $x_2(t)$, $x_3(t)$ для координаты ракеты, скорости и массы соответственно, учитывая, что в начальный момент ракета находится на земле, а $m(t) \hm \geqslant M > 0$ в каждый момент времени, мы можем переписать систему \eqref{eq2} при $t \in [0, T]$ в виде 
\hypertarget{p3}{}
\begin{equation} \label{eq3}
	\begin{cases} 
		\dot{x}_1(t) = x_2(t), \\
		\dot{x}_2(t) = -g + \dfrac{x_2(t) + l}{x_3(t)}u(t), \\
		\dot{x}_3(t) = -u(t), \\
		x_1(0) = 0, \\
		x_2(0) = 0, \\
		x_3(0) = m_0, \\
		x_2(T) \in [-\varepsilon, \varepsilon]. 
	\end{cases}
\end{equation}

Определим множество допустимых управлений как $\mathcal{P} = \{u(t) \! : [0, T] \rightarrow [0, u_{max}] \}$. В данной задаче необходимо найти программное управление $u(\cdot) \in \mathcal{P}$, которое переводит ракету на максимально возможную высоту, то есть минимизирует функционал 
\[\mathcal{J}(u(\cdot)) = \int\limits_{0}^{T}{-x_2(t)} \, \mathrm{d}t, \] на траекториях системы~\eqref{eq3}. 

\subsection{Принцип максимума Понтрягина}
Пусть $u^{*}(\cdot)$~--- оптимальное управление, $x^{*}(\cdot)$~--- оптимальная траектория. Тогда найдётся вектор-функция  $\psi(t) \! : [0, T] \rightarrow \mathbb{R}^4$, $\psi(t) \neq 0$ на $[0, T]$, которая удовлетворяет перечисленным ниже условиям.
\begin{enumerate}

\item Условию максимума
\hypertarget{p4}{}
\begin{equation}\label{eq4} 
\mathcal{H}(\psi, x^*, u^*) = \sup_{u(\cdot) \ \in \ \mathcal{P}}{\mathcal{H}(\psi, x^*, u)},
\end{equation} 
где $\mathcal{H}$~--- функция Гамильтона-Понтрягина системы \eqref{eq3}.

\item Сопряженной системе
\hypertarget{p5}{}
\begin{equation}\label{eq5}
\dot{\psi}(t) = - \dfrac{\partial{\mathcal{H}(\psi, x^*, u^*)}}{\partial{\bar{x}}},
\end{equation}
в которой за $x_0(t)$ обозначим $\int\limits_{0}^{t}{-x_2(\tau)} \, \mathrm{d}\tau$, а за $\bar{x}(t)$~--- вектор $x(t)$ с добавленной нулевой компонентой.

\item А также
\hypertarget{p7}{}
\begin{equation}\label{eq7}
	\begin{cases}
		\psi_0(t) = \mathrm{const} \leqslant 0, \\
		\sup\limits_{u(\cdot) \ \in \ \mathcal{P}}{\mathcal{H}(\psi, x^*, u)} \equiv \mathrm{const}.
	\end{cases}
\end{equation} 

\end{enumerate}

\subsection{Анализ поведения траекторий, подозрительных на оптимальность}
Выведем из условия максимума \eqref{eq4} значения оптимального управления в зависимости от параметров системы. Функция Гамильтона-Понтрягина для системы \eqref{eq3} запишется как 
\[ \mathcal{H}(\psi, x, u) = \psi_0(-x_2) + \psi_1x_2 + \psi_2\left(-g + \dfrac{x_2 + l}{x_3}u\right) - \psi_3u = \]
\[ = (\psi_1 - \psi_0)x_2 - \psi_2 \cdot g + \left(\dfrac{x_2 + l}{x_3}\psi_2 - \psi_3\right)u,\]
следовательно, учитывая ограничения на управление, получим
$$u^*(t) = \begin{cases}
	0, \ \ \ \dfrac{x_2(t) + l}{x_3(t)}\psi_2(t) - \psi_3(t) < 0 \\
	
[0, u_{max}], \ \ \ \dfrac{x_2(t) + l}{x_3(t)}\psi_2(t) - \psi_3(t) = 0 \\
	u_{max}, \ \ \ \dfrac{x_2(t) + l}{x_3(t)}\psi_2(t) - \psi_3(t) > 0.
\end{cases}$$
Рассмотрим сопряжённую систему \eqref{eq5}:
$$\begin{cases}
	\dot{\psi}_0(t) = - \dfrac{\partial{\mathcal{H}(\psi, x, u)}}{\partial{x_0}} = 0, \\
	\dot{\psi}_1(t) = - \dfrac{\partial{\mathcal{H}(\psi, x, u)}}{\partial{x_1}} = 0, \\
	\dot{\psi}_2(t) = - \dfrac{\partial{\mathcal{H}(\psi, x, u)}}{\partial{x_2}} = \psi_{0}(t) - \psi_{1}(t) - \dfrac{\psi_{2}(t)u(t)}{x_{3}(t)}, \\
	\dot{\psi}_3(t) = - \dfrac{\partial{\mathcal{H}(\psi, x, u)}}{\partial{x_3}} = \psi_2(t)u(t) \dfrac{(x_2(t) + l)}{x_3^2(t)}. \\
\end{cases} \Rightarrow$$

$$ \Rightarrow \begin{cases}
	\psi_0(t) = \psi_0^0 = \mathrm{const}, \\
	\psi_1(t) = \psi_1^0 = \mathrm{const}, \\
	\dot{\psi}_2(t) = \psi_{0}^0 - \psi_{1}^0 - \dfrac{\psi_{2}(t)u(t)}{x_{3}(t)}, \\
	\dot{\psi}_3(t) = \psi_2(t)u(t) \dfrac{(x_2(t) + l)}{x_3^2(t)}. \\
\end{cases}$$
Перейдём к качественному анализу траекторий.
\begin{itemize}
\item \textbf{Нормальный случай} \\
Без ограничения общности будем считать, что $\psi_0^0 = -1$. Так как ракета двигаться под землю не может, стартовое значение параметров должно быть таковым, чтобы $\dfrac{x_2(t) + l}{x_3(t)}\psi_2(t) -  \psi_3(t) \geqslant 0$, причём для взлёта необходимо $\dfrac{l \cdot u(0)}{m_0} > g$. Рассмотрим случай, когда ракета начинает своё движение со значения $u(t) = u_{max}$. Далее возможны следующие варианты.
\begin{enumerate}
\item Ракета пролетит всё время без переключений. \\
В данном случае траектории ракеты будут описываться системой
$$\begin{cases}
	\dot{x}_1(t) = x_2(t), \\
	\dot{x}_2(t) = -g + \dfrac{x_2(t) + l}{x_3(t)}u_{max}, \\
	\dot{x}_3(t) = -u_{max}, \\
	x_1(0) = 0, \\ 
	x_2(0) = 0, \\
	x_3(0) = m_0, \\
	x_2(T) \in [-\varepsilon, \varepsilon]. 
\end{cases} $$
Заметим, что тут скорость не убывает (в начальный момент $\dot{x}_2 > 0$ и дальше только увеличивается), а значит будет нарушено условие $ x_2(T) \in [-\varepsilon, \varepsilon] $, следовательно, система не имеет решений.

\item Произойдёт переключение на особый режим в момент времени $\tau_1^{\text{пер}}$. \\
Возможны два случая: 

	\begin{enumerate}
		\item Особый режим в течение $[\tau_1^{\text{пер}}, \tau_1^{\text{пер}} + \delta)$, где $\delta > 0$~--- некоторая константа. \\
		В особом режиме оптимальная траектория ракеты должна подчиняться следующим дифференциальным уравнениям:
$$\begin{cases}
	\dot{x}_1(t) = x_2(t), \\
	\dot{x}_2(t) = -g + \dfrac{x_2(t) + l}{x_3(t)}u^*(t), \\
	\dot{x}_3(t) = -u^*(t), \\
	\psi_0(t) = \psi_0^0, \\ 
	\psi_1(t) = \psi_1^0, \\
	\dot{\psi}_2(t) = \psi_{0}^0 - \psi_{1}^0 - \dfrac{\psi_{2}(t)}{x_{3}(t)}u^*(t), \\
	\dot{\psi}_3(t) = \psi_2(t) \dfrac{(x_2(t) + l)}{x_3^2(t)} u^*(t). \\ 
\end{cases} $$
При этом $ \dfrac{x_2(t) + l}{x_3(t)}\psi_2(t) - \psi_3(t) \equiv 0, \ t \in [\tau_1^{\text{пер}}, \tau_1^{\text{пер}} + \delta).$ 
Опираясь на это тождество, найдём выражение для управления, оставляющего нас в особом режиме. Для этого перейдём к соответствующим производным:
\[ \left(\dfrac{x_2(t) + l}{x_3(t)}\psi_2(t) - \psi_3(t) \right)' = \dfrac{\left(\dot{x}_2(t)\psi_2(t) + (x_2(t) + l)\dot{\psi}_2(t) \right) x_3(
t)}{x_3^2(t)} - \] 
\[ - \dfrac{\dot{x}_3(t)(x_2(t) + l)\psi_2(t)}{x_3^2(t)} - \dot{\psi}_3(t) \equiv 0, \]
что равносильно
\[\dfrac{\left( \left(-g + \dfrac{x_2(t) + l}{x_3(t)}u^*(t) \right)\psi_2(t) + \left(x_2(t) + l\right)\left(\psi_{0}^0 - \psi_{1}^0 - \dfrac{\psi_{2}(t)}{x_{3}(t)}u^*(t) \right) \right) x_3(t)}{x_3^2(t)} + \]
\[ + \dfrac{(x_2(t) + l)\psi_2(t)}{x_3^2(t)}u^*(t) - \psi_2(t) \dfrac{(x_2(t) + l)}{x_3^2(t)} u^*(t) = \dfrac{\left(\psi_{0}^0 - \psi_{1}^0\right)(x_2(t) + l) - g\psi_2(t)}{x_3(t)} \equiv 0 \]
и в конечном счёте сводится к равенству
\hypertarget{p8}{}
\begin{equation} \label{eq8}
\left(\psi_{0}^0 - \psi_{1}^0\right)(x_2(t) + l) = g\psi_2(t),  \ \ \forall t \in [\tau_1^{\text{пер}}, \tau_1^{\text{пер}} + \delta). 
\end{equation}
Чтобы получить значение необходимого управления, продифференцируем только что полученное тождество при $t \in [\tau_1^{\text{пер}}, \tau_1^{\text{пер}} + \delta)$: 
\begin{multline*}
\left(\psi_{0}^0 - \psi_{1}^0\right)\dot{x}_2(t) \equiv g\dot{\psi}_2(t) \Leftrightarrow \\
\Leftrightarrow  \left(\psi_{0}^0 - \psi_{1}^0\right)\left( -g + \dfrac{x_2(t) + l}{x_3(t)}u^*(t) \right) \equiv g\left( \psi_{0}^0 - \psi_{1}^0 - \dfrac{\psi_{2}(t)}{x_{3}(t)}u^*(t) \right) \Leftrightarrow \\
\Leftrightarrow 2g\left(\psi_{0}^0 - \psi_{1}^0\right) \equiv \left( \dfrac{\left(\psi_{0}^0 - \psi_{1}^0\right)(x_2(t) + l)}{x_3(t)} + g\dfrac{\psi_{2}(t)}{x_{3}(t)}\right)u^*(t) \Leftrightarrow \\
\Leftrightarrow \{ \left(\psi_{0}^0 - \psi_{1}^0\right)(x_2(t) + l) \equiv g\psi_2(t), \ \ t \in [\tau_1^{\text{пер}}, \tau_1^{\text{пер}} + \delta)\} \Leftrightarrow \\
\Leftrightarrow \psi_2(t)u^*(t) \equiv \left(\psi_{0}^0 - \psi_{1}^0\right) x_3(t).
\end{multline*}
Если $\psi_2(t)$ обнуляется хоть в какой-то момент в особом режиме, то $\psi_0^0 = \psi_1^0$, ведь $x_3(t) > 0$ при всех $t$. Разберем этот случай отдельно. 
\[ ] \ \psi_{0}^0 = \psi_{1}^0 \Rightarrow \{ \left(\psi_{0}^0 - \psi_{1}^0\right)(x_2(t) + l) = g\psi_2(t) \} \Rightarrow \psi_2(t) \equiv 0, \ \ t \in [\tau_1^{\text{пер}}, \tau_1^{\text{пер}} + \delta) \Rightarrow \] 
\[ \Rightarrow \bigg{\{} \dot{\psi}_2(t) = \psi_{0}^0 - \psi_{1}^0 - \dfrac{\psi_{2}(t)}{x_{3}(t)}u(t) \bigg{\}} \Rightarrow \psi_2(t) = 0, \ \ \forall t \in [0, T]\Rightarrow \] 
\[ \Rightarrow \bigg{\{} \dfrac{x_2(t) + l}{x_3(t)}\psi_2(t) = \psi_3(t) \bigg{\}} \Rightarrow \psi_3(t) = 0, \ \ t \in [\tau_1^{\text{пер}}, \tau_1^{\text{пер}} + \delta) \Rightarrow \]
\[ \Rightarrow \bigg{\{} \dot{\psi}_3(t) = \psi_2(t) \dfrac{(x_2(t) + l)}{x_3^2(t)} u(t) \bigg{\}} \Rightarrow \psi_3(t) = 0, \ \ \forall t \in [0, T]. \] 
Мы получили, что будем двигаться с константным вектором
$$\boxed{\psi(t) \equiv 
\begin{bmatrix}
-1 \\
-1 \\
0 \\
0
\end{bmatrix}}$$
на рассматриваемом промежутке времени. 

Теперь будем считать, что при $t \in [\tau_1^{\text{пер}}, \tau_1^{\text{пер}} + \delta)$ вектор $\psi_2(t) \neq 0.$ А значит,
$$ u^*(t) = \dfrac{(\psi_0^0 - \psi_1^0)x_3(t)}{\psi_2(t)}.$$
При этом должно выполняться $ 0 \leqslant u^*(t) \leqslant u_{max}$, поэтому знаки $\psi_2(t)$ и $\psi_0^0 - \psi_1^0$ должны совпадать ($x_3(t)$ всегда больше нуля), кроме того из проведенных до этого рассуждений следует, что обнуляются они также вместе.\\
Проанализируем, как меняются параметры системы в особом режиме. \\
Подставив полученное выражение для $u^*(t)$ в уравнение для производной $\psi_2(t)$, получим
\[\dot{\psi}_2(t) = 0, \ \forall t \in [\tau_1^{\text{пер
}}, \tau_1^{\text{пер}} + \delta) \Rightarrow \{ \left(\psi_{0}^0 - \psi_{1}^0\right)\dot{x}_2(t) \equiv g\dot{\psi}_2(t) \} \Rightarrow \]
$$\Rightarrow \left[ \begin{gathered}
\dot{x}_2(t) = 0, \ \forall t \in [\tau_1^{\text{пер}}, \tau_1^{\text{пер}} + \delta) \\
\psi_{0}^0 = \psi_{1}^0.
\end{gathered} \right. \Rightarrow $$ 
\[ \Rightarrow \{\text{случай $\psi_{0}^0 = \psi_{1}^0$ разобран выше} \} \Rightarrow \ \dot{x}_2(t) = 0, \ \forall t \in [\tau_1^{\text{пер}}, \tau_1^{\text{пер}} + \delta).\]
Заметим, что выражение $\dfrac{(x_2(\tau_1^{\text{пер}}) + l)}{x_3(\tau_1^{\text{пер}})}  >0$, приняв вместе с этим во внимание, что скорость не меняется в особом режиме, а $x_3(t)$ всегда положительно, из равенства $\dfrac{x_2(t) + l}{x_3(t)}\psi_2(t) = \psi_3(t)$ легко видеть, что $\psi_2(t)$ и $\psi_3(t)$ имеют одинаковый знак и обнуляются одновременно при $t \in [\tau_1^{\text{пер}}, \tau_1^{\text{пер}} + \delta)$. 
\begin{enumerate}
\item Предположим $(\psi_0^0 - \psi_1^0) > 0$:
\[ \dot{\psi}_3(t) > 0 \Leftrightarrow \psi_2(t) \dfrac{(x_2(t) + l)}{x_3^2(t)} u^*(t) > 0 \Leftrightarrow \bigg{\{} u^*(t) = \dfrac{(\psi_0^0 - \psi_1^0)x_3(t)}{\psi_2(t)} > 0 \bigg{\}} \Leftrightarrow \]
\[ \Leftrightarrow \psi_2(t) > 0 \Leftrightarrow \psi_3(t) > 0 \Leftrightarrow (\psi_0^0 - \psi_1^0) > 0,\]
для $t \in [\tau_1^{\text{пер}}, \tau_1^{\text{пер}} + \delta).$ \\ 
Из вышесказанного можно сделать вывод о том, что при полученных ограничениях на параметры системы и правильном управлении возможен случай $$\boxed{u_{max} \rightarrow u^* \text{ (эквивалентный перебору } \tau_1^{\text{пер}})}$$ 
Разберёмся с возможностями выхода из особого режима в момент времени $\tau_2^{\text{пер}}$. Так как $(\psi_0^0 - \psi_1^0) > 0$, следовательно, $\psi_2(\tau_2^{\text{пер}}) > 0$. Допустим произошло переключение на $u_{max}$. Тогда производная $\psi_2(t)$ станет изменяться по закону: 
\[ 	\dot{\psi}_2(t) = \psi_{0}^0 - \psi_{1}^0 - \dfrac{\psi_{2}(t)}{x_{3}(t)}u_{max}.\]
Покажем, что до следующего переключения $\dot{\psi}_2(t) \leqslant 0$. \\
\[ \left(\dfrac{\psi_2(t)}{x_3(t)}\right)' = \dfrac{\dot{\psi}_2(t)x_3(t) - \dot{x}_3(t)\psi_2(t)}{x_3^2(t)} = \dfrac{(\psi_{0}^0 - \psi_{1}^0)x_3(t)}{x_3^2(t)} > 0 \Rightarrow \]
\[ \Rightarrow \psi''_2(t) = -\left(\dfrac{\psi_2(t)}{x_3(t)}\right)' u_{max} < 0.\]
Кроме того, $\psi_2(\tau_2^{\text{пер}}) > 0$, $ u^*(t) \leqslant u_{max}$, а $\dot{\psi}_2(t) = 0$ в особом режиме, то есть $\dot{\psi}_2(\tau_2^{\text{пер}}) \leqslant 0$. Собирая всё в кучу, получаем что и хотели доказать: до следующего переключения $\dot{\psi}_2(t) \leqslant 0.$ \\
Теперь взглянем на то, как ведёт себя скорость
\[ \dot{x}_2(t) = -g + \dfrac{x_2(t) + l}{x_3(t)}u_{max}. \]
Она увеличивается, ведь до особого режима $\dot{x}_2(t) > 0$, в особом $\dot{x}_2(t) \hm = 0$. \\
Выход на максимальное управление подразумевает под собой 
\[\left.\left(\dfrac{x_2(t) + l}{x_3(t)}\psi_2(t) - \psi_3(t)\right)' \right|_{t = \tau_2^{\text{пер}}} > 0 \Leftrightarrow \{ \text{равенство } \eqref{eq8}\} \Leftrightarrow \]
\[\Leftrightarrow \left(\psi_{0}^0 - \psi_{1}^0\right)(x_2(\tau_2^{\text{пер}}) + l) > g\psi_2(\tau_2^{\text{пер}}). \]
Обратим внимание, на полученные выше результаты, а именно: $\dot
{x}_2(t) > 0$, $\dot{\psi}_2(t) \leqslant 0$. Тут становится очевидным, что 
\[ \left(\dfrac{x_2(t) + l}{x_3(t)}\psi_2(t) - \psi_3(t)\right)' > 0 \] 
до конца текущего режима. Это говорит нам об отсутствии дальнейших переключений (возможно только по израсходованию всего топлива). Таким образом получили ситуацию из пункта 1., которая является неразрешимой, следовательно рано или поздно должно закончиться топливо и начаться торможение.\\
Проведёнными выше рассуждениями мы показали, что оптимальная траектория может определяться следующей последовательностью переключений управления: $$\boxed{u_{max} \rightarrow u^* \rightarrow u_{max} \rightarrow 0 \ \ \  (\tau_1^{\text{пер}} \rightarrow \tau_2^{\text{пер}} \rightarrow \tau_3^{\text{пер}} \text{ соответственно})}$$
Теперь предположим, что из особого режима произошло переключение на нулевое управление. Докажем, что в таком случае переключений больше не произойдёт. При текущем развитии событий $\dot{x}_2(t) = -g < 0$, $\dot{\psi}_2(t) \hm = \psi_0^0 - \psi_1^0 > 0$. Учитывая, что 
\[\left.\left(\dfrac{x_2(t) + l}{x_3(t)}\psi_2(t) - \psi_3(t)\right)' \right|_{t = \tau_2^{\text{пер}}} < 0 \Leftrightarrow \{ \text{равенство } \eqref{eq8}\} \Leftrightarrow \]
\[ \Leftrightarrow \left(\psi_{0}^0 - \psi_{1}^0\right)(x_2(\tau_2^{\text{пер}}) + l) < g\psi_2(\tau_2^{\text{пер}}), \]
получаем 
\[ \left(\dfrac{x_2(t) + l}{x_3(t)}\psi_2(t) - \psi_3(t)\right)' < 0 \] 
до конца, отсюда следует, что мы хотели доказать. А значит возможна последовательность 
$$\boxed{u_{max} \rightarrow u^* \rightarrow 0 \ \ \  (\tau_1^{\text{пер}} \rightarrow \tau_2^{\text{пер}} \text{ соответственно})}$$ 
\item Если же $(\psi_0^0 - \psi_1^0) < 0$, то проделав все те же рассуждения, что и для предыдущего случая, мы придём к возможности колебаний режимов управления (нет гарантии стабилизации в фиксированном режиме). В данной ситуации будем из системы 
$$\begin{cases}
	\dfrac{x_2(\tau_1^{\text{пер}}) + l}{x_3(\tau_1^{\text{пер}})}\psi_2(\tau_1^{\text{пер}}) - \psi_3(\tau_1^{\text{пер}}) = 0, \\
	\dfrac{x_2(\tau_2^{\text{пер}}) + l}{x_3(\tau_2^{\text{пер}})}\psi_2(\tau_2^{\text{пер}}) - \psi_3(\tau_2^{\text{пер}}) = 0
\end{cases}$$
определять значения $\psi_2^0$, $\psi_3^0$ в зависимости от $\tau_1^{\text{пер}}$, $\tau_2^{\text{пер}}$ и по ним уже численно строить подозрительные траектории, интегрируя всю гамильтонову систему. Значения $x_2(t)$ и $x_3(t)$ в моменты $\tau_1^{\text{пер}}$, $\tau_2^{\text{пер}}$ несложно получить, так как мы знаем уравнения их изменения и при $u_{max}$, и когда $u^*(t) = \dfrac{g x_3(t)}{x_2(t) + l}$, а также начальные условия. Функции $\psi_2(\psi_2^0, t)$, $\psi_3(\psi_3^0, t)$ предлагается определять при помощи символьных вычислений в $\texttt{matlab}$ или $\texttt{wolfram}$.
\end{enumerate}

\item Сразу же вышли из особого режима. \\
Обратно на $u_{max}$~--- данный случай ничем не отличается от пункта 1. Если же мы переключились на нулевое управление, то $\dot{x}_2(t) = -g$, $\dot{\psi}_2(t) = \psi_0^0 \hm - \psi_1^0$. Посмотрим на знак производной функции, определяющей оптимальное в данный момент управление.
\[ \left(\dfrac{x_2(t) + l}{x_3(t)}\psi_2(t) - \psi_3(t)\right)' \vee 0 \Leftrightarrow \left(\psi_{0}^0 - \psi_{1}^0\right)(x_2(t) + l) \vee g\psi_2(t) \Rightarrow  \] 
\[ \Rightarrow \{ \text{переходя к производным}\} \Rightarrow 0 \vee (\psi_{0}^0 - \psi_{1}^0). \]
В момент выхода производная была меньше либо равна нуля, а значит указанное переключение сохраняется когда $(\psi_{0}^0 - \psi_{1}^0) \geqslant 0$. Данный случай сводится к уже разбиравшемуся выше 
$$\boxed{u_{max} \rightarrow u^* \rightarrow 0$ (где $\tau_1^{\text{пер}} = \tau_2^{\text{пер}} \rightarrow \tau_3^{\text{пер}})}$$ 
Для $(\psi_0^0 - \psi_1^0) < 0$ разобьём на два подслучая: 
\begin{itemize}[label=$\ast$]
\item $u_{max} \rightarrow 0$.\\
Данный вариант уже был представлен к рассмотрению ранее.
\item $u_{max} \rightarrow 0 \rightarrow \dots$ \\\
Действуем как и в пункте $\mathrm{ii}$ с учётом того, что второй режим нулевой, а не особый.
\end{itemize}
\end{enumerate}

\item Добавим в разобранные случаи выше возможность израсходовать топливо (где это явно не указывалось). \\
Здесь мы качественно новых случаев не получим. Единственное, что стоит отметить, так это то, что при численном решении задачи необходимо будет отдельно следить за расходом топлива и производить переключение при его израсходовании.
\end{enumerate} 
Если рассмотреть старт из особого режима, качественно новых случаев мы не получим (появившиеся сводятся к имеющимся: $0 = \tau_1^{\text{пер}} \rightarrow \tau_2^{\text{пер}} \rightarrow \tau_3^{\text{пер}}$).

\item \textbf{Анормальный случай} \\
Пусть $\psi_0^0 = 0$. Тогда несложно увидеть, что анормальный случай~--- частный случай (качественно) нормального режима. 

\item \textbf{Итог} \\
Подытожив всё вышесказанное, нетрудно видеть, что задача сводится к обработке трёх возможностей:  
\begin{enumerate}
	\item $u_{max} \rightarrow 0$. \\
Ведём перебор $\tau_1^{\text{пер}} \in (0, T)$. Для каждого рассматриваемого значения, используя ранее полученные уравнения поведения ракеты при $u_{max}$ и $0$, находим подозрительные траектории.
	\item $u_{max} \rightarrow u^* \rightarrow \dots$ \\
Строим орбиты согласно пункту $\mathrm{ii}$ с сеткой $0 \leqslant \tau_1^{\text{пер}} < \tau_2^{\text{пер}} \leqslant T$.
	\item $u_{max} \rightarrow 0 \rightarrow \dots$ \\
Интегрирование происходит аналогично предыдущему пункту, с разницей в уравнениях второго режима.
\end{enumerate}
Далее для всех найденных траекторий проверяется краевое условие $x_2(T) \in [-\varepsilon, \varepsilon]$ и вычисляется функционал. Возвращаемое оптимальное решение выбирается путём минимизирования значения функционала на орбитах, приводящих нас в заданное конечное множество. 

Остаётся только избавиться от случая константного вектора $\psi(t) \equiv [-1, -1, 0, 0]^{T}$. Ведь для него мы получаем особый режим на протяжении всего $[0, T]$ без явно заданной формулы оптимального управления. То есть теоретически оптимальное решение может соответствовать именно этому сопряженному вектору, в то время как численное решение отталкиваясь от него мы построить не сможем за неимением закона изменения управления.
\end{itemize} 
Заметим, что переменная высоты ($x_1(t)$) не участвует в выражении для оптимального управления, функционале, сопряженной системе, уравнениях для других фазовых переменных, к тому же однозначно восстанавливается по скорости, а значит от неё можно избавиться, сократив при этом как фазовый, так и сопряженный вектор, сведя задачу к более простой. Подчеркнём, что по оптимальному решению упрощенной задачи мы всегда сможем восстановить лучшее решение исходной. Интегрируемая система примет вид: 
$$\begin{cases}
	\dot{x}_2(t) = -g + \dfrac{x_2(t) + l}{x_3(t)}u^*(t), \\
	\dot{x}_3(t) = -u^*(t), \\
	\psi_0(t) = \psi_0^0, \\ 
	\dot{\psi}_2(t) = \psi_{0}^0 - \dfrac{\psi_{2}(t)}{x_{3}(t)}u^*(t), \\
	\dot{\psi}_3(t) = \psi_2(t) \dfrac{(x_2(t) + l)}{x_3^2(t)} u^*(t). \\ 
\end{cases} $$
Не представляет сложности убедиться в том, что мы получили ту же систему только с $\psi_1^0 = 0$. Поэтому для текущей будут верны все проделанные ранее выкладки с учётом значения $\psi_0^0 \hm = (\psi_0^0 - \psi_1^0)$. В нормальном случае $(\psi_0^0 - \psi_1^0) = -1 < 0$, когда постоянный вектор $\psi(t)$ соответствовал $\psi_0^0 = \psi_1^0 = -1$, из-за этого для данной ситуации он не представляется возможным и значит оптимальное решение всё-таки можно построить без него. В анормальном режиме $\psi_0^0 = 0$, рассмотрим его подробнее. Если в анормальном режиме был особый, то из \eqref{eq8} мы получаем, что 
\[ \psi_2(t) = 0, \ \ t \in [\tau_1^{\text{пер}}, \tau_1^{\text{пер}} + \delta) \Rightarrow \] 
\[ \Rightarrow \bigg{\{} \dot{\psi}_2(t) = \psi_{0}^0  - \dfrac{\psi_{2}(t)}{x_{3}(t)}u(t) \bigg{\}} \Rightarrow \psi_2(t) = 0, \ \ \forall t \in [0, T]\Rightarrow \] 
\[ \Rightarrow \bigg{\{} \dfrac{x_2(t) + l}{x_3(t)}\psi_2(t) = \psi_3(t) \bigg{\}} \Rightarrow \psi_3(t) = 0, \ \ t \in [\tau_1^{\text{пер}}, \tau_1^{\text{пер}} + \delta) \Rightarrow \]
\[ \Rightarrow \bigg{\{} \dot{\psi}_3(t) = \psi_2(t) \dfrac{(x_2(t) + l)}{x_3^2(t)} u(t) \bigg{\}} \Rightarrow \psi_3(t) = 0, \ \ \forall t \in [0, T]. \]
Противоречие с принципом максимума Понтрягина ($\psi(t) \equiv 0$). Если же особого режима не было, то как было показано ранее для $(\psi_0^0 - \psi_1^0) = 0$ нулевое управление сохранится после переключения с $u_{max}$ (пришли к уже учтённому выше случаю $u_{max} \rightarrow 0$). Поэтому оптимальное решение будет лежать в одном из трёх представленных случаев.
\newpage

\section{Первое аналитическое решение второй задачи}
Исходная система~\eqref{eq1} для второй задачи при $t \in [0, T]$ имеет вид
\hypertarget{p9}{}
\begin{equation} \label{eq9}
	\begin{cases} 
		\dot{m}(t) v(t) + m(t)\dot{v}(t) = -gm(t) + lu(t), \\
		\dot{m}(t) = -u(t), \\
		m(0) = m_0, \\
		v(0) = 0.
		\end{cases}
\end{equation}

Вводя новые обозначения $x_1(t)$, $x_2(t)$, $x_3(t)$ для координаты ракеты, скорости и массы соответственно, учитывая, что в начальный момент ракета находится на земле, в конечный момент~--- на заданной высоте $H$, а $m(t) \geqslant M > 0$ в каждый момент времени, мы можем переписать систему~\eqref{eq9} при $t \in [0, T]$ в виде 
\hypertarget{p10}{}
\begin{equation} \label{eq10}
	\begin{cases} 
		\dot{x}_1(t) = x_2(t), \\
		\dot{x}_2(t) = -g + \dfrac{x_2(t) + l}{x_3(t)}u(t), \\
		\dot{x}_3(t) = -u(t), \\
		x_1(0) = 0, \\
		x_2(0) = 0, \\
		x_3(0) = m_0, \\
		x_1(T) = H. 
	\end{cases}
\end{equation}

В данной задаче необходимо найти программное управление $u(\cdot) \in \mathcal{P}$, которое переводит ракету на заданную высоту $H > 0$ в заданный момент времени $T > 0$ и при этом минимизирует функционал 
\[\mathcal{J}(u(\cdot)) = \int\limits_{0}^{T}{u^4(t)} \, \mathrm{d}t, \] на траекториях системы~\eqref{eq10}. 

Для удобства решения укоротим систему \eqref{eq10} при $t \in [0, T]$ до вида 
\hypertarget{p11}{}
\begin{equation} \label{eq11}
	\begin{cases} 
		\dot{x}_2(t) = -g + \dfrac{x_2(t) + l}{x_3(t)}u(t), \\
		\dot{x}_3(t) = -u(t), \\
		x_2(0) = 0, \\
		\int\limits_{0}^{T}{x_2(t)} \, \mathrm{d}t = H, \\
		x_3(0) = m_0. 
	\end{cases}
\end{equation}

Где условие $\int\limits_{0}^{T}{x_2(t)} \, \mathrm{d}t = H$ представляет собой изначальное требование $x_1(T) = H.$ Для решения задачи нам достаточно найти оптимальную траекторию укороченной системы, ведь $x_1(t)$ не входит в функционал и дифференциальные уравнения на другие переменные системы, а кроме того однозначно восстанавливается по $x_2(t).$

\subsection{Принцип максимума Понтрягина}
Пусть $u^{*}(\cdot)$~--- оптимальное управление, $x^{*}(\cdot)$~--- оптимальная траектория. Тогда найдётся вектор-функция  $\psi(t) \! : [0, T] \rightarrow \mathbb{R}^3$, $\psi(t) \neq 0$ на $[0, T]$, которая удовлетворяет перечисленным ниже условиям.
\begin{enumerate}

\item Условию максимума
\hypertarget{p12}{}
\begin{equation}\label{eq12} 
\mathcal{H}(\psi, x^*, u^*) = \sup_{u(\cdot) \ \in \ \mathcal{P}}{\mathcal{H}(\psi, x^*, u)},
\end{equation} 
где $\mathcal{H}$~--- функция Гамильтона-Понтрягина системы \eqref{eq11}.

\item Сопряженной системе
\hypertarget{p13}{}
\begin{equation}\label{eq13}
\dot{\psi}(t) = - \dfrac{\partial{\mathcal{H}(\psi, x^*, u^*)}}{\partial{\tilde{x}}},
\end{equation}
в которой за $x_0(t)$ обозначим $\int\limits_{0}^{t}{
u^4(\tau)} \, \mathrm{d}\tau$, а за $\tilde{x}(t)$~--- вектор $x(t)$ с добавленной нулевой компонентой.

\item А также
\hypertarget{p15}{}
\begin{equation}\label{eq15}
	\begin{cases}
		\psi_0(t) = \mathrm{const} \leqslant 0, \\
		\sup\limits_{u(\cdot) \ \in \ \mathcal{P}}{\mathcal{H}(\psi, x^*, u)} \equiv \mathrm{const}.
	\end{cases}
\end{equation} 

\end{enumerate}

\subsection{Анализ поведения траекторий, подозрительных на оптимальность} 
Запишем функцию Гамильтона-Понтрягина для системы \eqref{eq11} 
\[ \mathcal{H}(\psi, x, u) = \psi_0 u^4 +\psi_2\left(-g + \dfrac{x_2 + l}{x_3}u\right) - \psi_3u = \]
\[ = \psi_0 u^4 - \psi_2 g + \left(\dfrac{x_2 + l}{x_3}\psi_2 - \psi_3\right)u,\] 
за $a(\psi, x)$ обозначим выражение $\dfrac{x_2(t) + l}{x_3(t)}\psi_2(t) - \psi_3(t)$, а $c(\psi)$ примем равным $-\psi_2(t)g$. \\
Распишем сопряжённую систему \eqref{eq13}
$$\begin{cases}
	\dot{\psi}_0(t) = - \dfrac{\partial{\mathcal{H}(\psi, x, u)}}{\partial{x_0}} = 0, \\
	\dot{\psi}_2(t) = - \dfrac{\partial{\mathcal{H}(\psi, x, u)}}{\partial{x_2}} = - \dfrac{\psi_{2}(t)u(t)}{x_{3}(t)}, \\
	\dot{\psi}_3(t) = - \dfrac{\partial{\mathcal{H}(\psi, x, u)}}{\partial{x_3}} = \psi_2(t)u(t) \dfrac{(x_2(t) + l)}{x_3^2(t)}. \\
\end{cases} \Rightarrow$$

$$ \Rightarrow \begin{cases}
	\psi_0(t) = \psi_0^0 = \mathrm{const}, \\
	\dot{\psi}_2(t) = - \dfrac{\psi_{2}(t)u(t)}{x_{3}(t)}, \\
	\dot{\psi}_3(t) = \psi_2(t)u(t) \dfrac{(x_2(t) + l)}{x_3^2(t)}. \\
\end{cases}$$
Теперь приступим к качественному анализу траекторий.
\begin{itemize}
\item \textbf{Нормальный случай} \\
Без ограничения общности будем считать, что $\psi_0^0 = -1$. Тогда функция Гамильтона-Понтрягина системы примет вид:
\[\mathcal{H}(\psi, x, u) = -u^4 - c(\psi) + a(\psi, x)u.\]
Найдём из условия максимума \eqref{eq12} значения оптимального управления в зависимости от параметров системы, решив задачу максимизации функции $\mathcal{H}(u)$. Заметим, что наша функция непрерывно дифференцируема на выпуклом множестве $[0, u_{max}]$, в то же время является выпуклой на отрезке, а значит максимум будет достигаться в точках $u_*$, для которых выполнено вариационное неравенство 
\[ \mathcal{H}'(u_*) \cdot (u - u_*) \leqslant 0 \ \ \ \forall u \in [0, u_{max}] \Leftrightarrow \]
\[ \Leftrightarrow (-4u^3_* + a)(u - u_*) \leqslant 0 \ \ \ \forall u \in [0, u_{max}], \]
откуда несложно определяется оптимальное управление
$$u^*(t) = \begin{cases}
	0, \ \ \ a(t) \leqslant 0 \\
	\sqrt[3]{\dfrac{a(t)}{4}}, \ \ \ 0 \leqslant a(t) \leqslant 4u_{max}^3  \\
	u_{max}, \ \ \ a(t) \geqslant 4u_{max}^3.
\end{cases}$$

Так как ракета двигаться под землю не может, стартовое значение параметров должно быть таковым, чтобы $a(0) \geqslant 4 \left(\dfrac{m_0 g}{l} \right)^3 $. 
Исследуем качественное поведение оптимального управления системы. Для этого рассмотрим первую и вторую производные $a(t)$.
\[ \left(\dfrac{x_2(t) + l}{x_3(t)}\psi_2(t) - \psi_3(t) \right)' = \dfrac{\left(\dot{x}_2(t)\psi_2(t) + (x_2(t) + l)\dot{\psi}_2(t) \right) x_3(t)}{x_3^2(t)} - \] 
\[ - \dfrac{\dot{x}_3(t)(x_2(t) + l)\psi_2(t)}{x_3^2(t)} - \dot{\psi}_3(t) \vee 0, \]
что равносильно
\[\dfrac{\left( \left(-g + \dfrac{x_2(t) + l}{x_3(t)}u^*(t) \right)\psi_2(t) + \left(x_
2(t) + l\right)\left(- \dfrac{\psi_{2}(t)}{x_{3}(t)}u^*(t) \right) \right) x_3(t)}{x_3^2(t)} + \]
\[ + \dfrac{(x_2(t) + l)\psi_2(t)}{x_3^2(t)}u^*(t) - \psi_2(t) \dfrac{(x_2(t) + l)}{x_3^2(t)} u^*(t) = \dfrac{- g\psi_2(t)}{x_3(t)} \vee 0 \]
и в конечном счёте сводится к неравенству
\hypertarget{p16}{}
\begin{equation} \label{eq16}
- \psi_2(t) \vee 0,  \ \ \forall t \in [0, T]. 
\end{equation}
Определим вторую производную $a(t)$
\[ \left( \dfrac{- g\psi_2(t)}{x_3(t)} \right)' =  \dfrac{-g\dot{\psi}_2(t)x_3(t) + \dot{x}_3(t)g \psi_2(t)}{x_3^2(t)} = \dfrac{-g \left( - \dfrac{\psi_2(t)u^*(t)}{x_3(t)} \right)x_3(t) -u^*(t)g\psi_2(t)}{x_3^2(t)} = 0, \]
следовательно $a'(t) = const$ и $\psi_2(t)$ не меняет знак на всём рассматриваемом промежутке времени. Учитывая, что оптимальное управление имеет сопутствующий $a(t)$ характер изменения, легко сделать вывод о том, что подозрительные на оптимальность траектории можно разбить на два случая:

\begin{enumerate}
\item Управление возрастает по ходу движения. \\
Траектории в данном случае однозначно определяются двумя параметрами задающими прямую $a(t)$ ($a(0)$ и $\alpha$, к примеру, где $\alpha$~--- угол наклона). \\
Будем считать, что движение начинается с участка меняющегося управления, который соответствует указанной ниже системе при $t \in [0, \tau_1^{\text{пер}}]$ 
\hypertarget{p17}{}
\begin{equation} \label{eq17}
	\begin{cases} 
		\dot{x}_2(t) = -g + \dfrac{x_2(t) + l}{x_3(t)}u(t), \\
		\dot{x}_3(t) = -u(t), \\
		x_2(0) = 0, \\
		x_3(0) = m_0,
	\end{cases}
\end{equation}
за $\tau_1^{\text{пер}}$ обозначим первое переключение режима с переменным управлением на режим с постоянным управлением. \\
Управление при этом выражается следующим образом
\hypertarget{p18}{}
\begin{equation} \label{eq18}
 u^*(t) = \sqrt[3]{\dfrac{\alpha \cdot t + a(0) }{4}}. 
\end{equation}
Если топливо закончилось раньше, чем мы успели достигнуть $u_{max}$, то в момент времени $\tau_1^{\text{пер}}$ произойдёт переключение на нулевое управление, при котором полёт ракеты будет подчиняться следующей системе для $t \in [\tau_1^{\text{пер}}, T]$
\hypertarget{p19}{}
\begin{equation} 
	\begin{cases} \label{eq19}
		\dot{x}_2(t) = -g, \\
		\dot{x}_3(t) = 0, \\
		x_2(\tau_1^{\text{пер}}) = \bar{x}_2, \\
		x_3(\tau_1^{\text{пер}}) = \bar{x}_3,
	\end{cases}
\end{equation} 
где $\bar{x}_2$, $\bar{x}_3$~--- известные значения, полученные путём интегрирования системы \eqref{eq17}. Сам момент времени $\tau_1^{\text{пер}}$ будем определять для каждой рассматриваемой траектории посредством функции события в численной реализации решения. \\
Допустим мы вышли на $u_{max}$ раньше, чем запасы топлива иссякли, тогда движение ракеты с этого момента будет задаваться уравнениями
\hypertarget{p20}{}
\begin{equation} \label{eq20}
	\begin{cases} 
		\dot{x}_2(t) = -g + \dfrac{x_2(t) + l}{x_3(t)}u_{max}, \\
		\dot{x}_3(t) = -u_{max}, \\
		x_2(\tau_1^{\text{пер}}) = \tilde{x}_2, \\
		x_3(\tau_1^{\text{пер}}) = \tilde{x}_3,
	\end{cases}
\end{equation}

где $\tilde{x}_2$, $\tilde{x}_3$~--- известные значения, полученные путём интегрирования системы \eqref{eq18}. Переключение на $u_{max}$ также отлавливаем посредством функции событий ($a(t) = 4u^3_{max}$). \\
Далее либо мы до конца летим на $u_{max}$, либо в определённый момент закончится топливо. \\
Таким образом, мы разобрали все возможные для данного случая варианты, которые, ещё раз подчеркнём, сводятся к перебору указанных в начале параметров.

\item Управление не возрастает по ходу движения. \\
Траектории в данном случае тоже однозначно определяются двумя параметрами, задающими прямую $a(t)$. \\
Можем считать, что движение стартует с участка переменного управления, на котором траектории подчиняются системе \eqref{eq17}, управление задаётся формулой~\eqref{eq18}. \\
Далее при переключении на нулевое управление полёт станет описываться дифференциальными уравнениями \eqref{eq19}. Переход на нулевое управление будем фиксировать через функции событий. \\
Не забудем про продолжительный старт с $u_{max}$, для этого станем перебирать $0 < \tau_1^{\text{пер}} < T$ ~--- момент перехода с $u_{max}$, а также угол $\alpha \in (-\frac{\pi}{2}, 0)$. \\
Отметим, что случай движения всё время с переменным управлением также будет рассмотрен в представленном решении.
\end{enumerate}

\item \textbf{Анормальный случай} \\
Пусть $\psi_0^0 = 0$. Управление из условия максимума \eqref{eq12} определяется как
$$u^*(t) = \begin{cases}
	0, \ \ \ a(t) < 0 \\
	[0, u_{max}], \ \ \ a(t) = 0 \\
	u_{max}, \ \ \ a(t) > 0.
\end{cases}$$
При выводе $a''(t)$ мы не пользовались явным видом управления, а значит $a'(t) = \texttt{const}$ верно и для анормального режима, из чего следует, что в нём особый режим невозможен. Так как стартовать с нулевого управления мы не можем, остаются два возможных варианта движения ракеты в текущем режиме:
\begin{enumerate}
\item Полёт на $u_{max}$ без переключений. \\
Этот вариант учтён в нормальном случае, поэтому он нам не интересен.
\item Стартуем с $u_{max}$, а затем происходит переключение на 0 в момент времени $\tau_1^{\text{пер}}$. \\
Траектории здесь однозначно определяются переходом на нулевое управление. 
\end{enumerate}
Таким образом, достаточно рассмотреть лишь перебор $0 < \tau_1^{\text{пер}} < T$.

\item \textbf{Итог} \\
Подытожив всё вышесказанное, можно утверждать, что решение задачи определяется следующими переборами параметров: $a(0) \in [4 \left(\frac{m_0 g}{l} \right)^3, 4u^3_{max}]$, $\alpha \in (-\frac{\pi}{2}, \frac{\pi}{2})$, момента $0 \hm < \tau_1^{\text{пер}} < T$ с $\alpha \in (-\frac{\pi}{2}, 0)$ и отдельно $0 < \tau_1^{\text{пер}} < T$. Для каждой построенной траектории также необходимо сделать проверку на выполнение условия $\int\limits_{0}^{T}{x_2(t)} \, \mathrm{d}t = H$. 
\end{itemize}
\newpage 

\section{Второе аналитическое решение второй задачи}
Проговорим минусы первого решения:
\begin{enumerate}
\item Перебор по начальным значениям, вместо перебора по моментам переключения. \\
Это значительно понижает точность, из-за быстро растущей со временем погрешности численного решения (скорость зависит от используемых функций, но для любых методов является сильно нелинейной).
\item Проверка условия $\int\limits_{0}^{T}{x_2(t)} \, \mathrm{d}t = H$. \\ Непрактична из-за неотвратимых приближений, присущих программной реализации.
\end{enumerate}
Ввиду чего в новом решении попробуем отталкиваться от условий трансверсальности, учитывающих $x_1(T) = H$ и при переборе стараться опираться именно на моменты смены управления.

\subsection{Принцип максимума Понтрягина}
Пусть $u^{*}(\cdot)$~--- оптимальное управление, $x^{*}(\cdot)$~--- оптимальная траектория. Тогда найдётся вектор-функция  $\psi(t) \! : [0, T] \rightarrow \mathbb{R}^4$, $\psi(t) \neq 0$ на $[0, T]$, которая удовлетворяет перечисленным ниже условиям.
\begin{enumerate}

\item Условию максимума
\hypertarget{p21}{}
\begin{equation}\label{eq21} 
\mathcal{H}(\psi, x^*, u^*) = \sup_{u(\cdot) \ \in \ \mathcal{P}}{\mathcal{H}(\psi, x^*, u)},
\end{equation} 
где $\mathcal{H}$~--- функция Гамильтона-Понтрягина системы \eqref{eq10}.

\item Сопряженной системе
\hypertarget{p22}{}
\begin{equation}\label{eq22}
\dot{\psi}(t) = - \dfrac{\partial{\mathcal{H}(\psi, x^*, u^*)}}{\partial{\tilde{x}}},
\end{equation}
в которой за $x_0(t)$ обозначим $\int\limits_{0}^{t}{u^4(\tau)} \, \mathrm{d}\tau$, а за $\tilde{x}(t)$~--- вектор $x(t)$ с добавленной нулевой компонентой.

\item Условию трансверсальности
\hypertarget{p23}{}
\begin{equation}\label{eq23}
	\begin{cases}
		\bar{\psi}(t_0) \perp T_{x^0}\mathcal{X}^0, \\
		\bar{\psi}(t_1) \perp T_{x^1}\mathcal{X}^1,
	\end{cases}
\end{equation} 
где $T_{x^0}\mathcal{X}^0$~--- касательная гиперплоскость к начальной точке, $T_{x^1}\mathcal{X}^1$~--- касательная гиперплоскость к конечному множеству, а $\bar{\psi}(t)$~--- сопряжённый вектор без нулевой компоненты.

\item А также
\hypertarget{p24}{}
\begin{equation}\label{eq24}
	\begin{cases}
		\psi_0(t) = \mathrm{const} \leqslant 0, \\
		\sup\limits_{u(\cdot) \ \in \ \mathcal{P}}{\mathcal{H}(\psi, x^*, u)} \equiv \mathrm{const}.
	\end{cases}
\end{equation} 

\end{enumerate}



\subsection{Анализ поведения траекторий, подозрительных на оптимальность} 
Запишем функцию Гамильтона-Понтрягина для системы \eqref{eq10} 
\[ \mathcal{H}(\psi, x, u) = \psi_0 u^4 + \psi_1 x_2 + \psi_2\left(-g + \dfrac{x_2 + l}{x_3}u\right) - \psi_3u = \]
\[ = \psi_0 u^4 + \psi_1 x_2 - \psi_2 g + \left(\dfrac{x_2 + l}{x_3}\psi_2 - \psi_3\right)u,\] 
за $a(\psi, x)$ обозначим выражение $\dfrac{x_2(t) 
+ l}{x_3(t)}\psi_2(t) - \psi_3(t)$, а $c(\psi)$ примем равным $\psi_1 x_2 -\psi_2(t)g$. \\
Распишем сопряжённую систему \eqref{eq22}
$$\begin{cases}
	\dot{\psi}_0(t) = - \dfrac{\partial{\mathcal{H}(\psi, x, u)}}{\partial{x_0}} = 0, \\
	\dot{\psi}_1(t) = - \dfrac{\partial{\mathcal{H}(\psi, x, u)}}{\partial{x_1}} = 0, \\
	\dot{\psi}_2(t) = - \dfrac{\partial{\mathcal{H}(\psi, x, u)}}{\partial{x_2}} = -\psi_1^0 - \dfrac{\psi_{2}(t)u(t)}{x_{3}(t)}, \\
	\dot{\psi}_3(t) = - \dfrac{\partial{\mathcal{H}(\psi, x, u)}}{\partial{x_3}} = \psi_2(t)u(t) \dfrac{(x_2(t) + l)}{x_3^2(t)}. \\
\end{cases} \Rightarrow$$

$$ \Rightarrow \begin{cases}
	\psi_0(t) = \psi_0^0 = \mathrm{const}, \\
	\psi_1(t) = \psi_1^0 = \mathrm{const}, \\
	\dot{\psi}_2(t) = - \psi_1^0 - \dfrac{\psi_{2}(t)u(t)}{x_{3}(t)}, \\
	\dot{\psi}_3(t) = \psi_2(t)u(t) \dfrac{(x_2(t) + l)}{x_3^2(t)}. \\
\end{cases}$$
Теперь приступим к качественному анализу траекторий.
\begin{itemize}
\item \textbf{Нормальный случай} \\
Без ограничения общности будем считать, что $\psi_0^0 = -1$. Тогда функция Гамильтона-Понтрягина системы примет вид:
\[\mathcal{H}(\psi, x, u) = -u^4 - c(\psi) + a(\psi, x)u.\]
Найдём из условия максимума \eqref{eq21} значения оптимального управления в зависимости от параметров системы, решив задачу максимизации функции $\mathcal{H}(u)$. Заметим, что наша функция непрерывно дифференцируема на выпуклом множестве $[0, u_{max}]$, в то же время является выпуклой на отрезке, а значит максимум будет достигаться в точках $u_*$, для которых выполнено вариационное неравенство 
\[ \mathcal{H}'(u_*) \cdot (u - u_*) \leqslant 0 \ \ \ \forall u \in [0, u_{max}] \Leftrightarrow \]
\[ \Leftrightarrow (-4u^3_* + a)(u - u_*) \leqslant 0 \ \ \ \forall u \in [0, u_{max}], \]
откуда несложно определяется оптимальное управление
\hypertarget{p25}{}
\begin{equation} \label{eq25}
	u^*(t) = \begin{cases} 
		0, \ \ \ a(t) \leqslant 0 \\
		\sqrt[3]{\dfrac{a(t)}{4}}, \ \ \ 0 \leqslant a(t) \leqslant 4u_{max}^3  \\
		u_{max}, \ \ \ a(t) \geqslant 4u_{max}^3.
	\end{cases}
\end{equation}

Так как ракета двигаться под землю не может, стартовое значение параметров должно быть таковым, чтобы $a(0) \geqslant 4 \left(\dfrac{m_0 g}{l} \right)^3 $. Исследуем качественное поведение оптимального управления системы. Для этого рассмотрим первую и вторую производные $a(t)$.
\[ \left(\dfrac{x_2(t) + l}{x_3(t)}\psi_2(t) - \psi_3(t) \right)' = \dfrac{\left(\dot{x}_2(t)\psi_2(t) + (x_2(t) + l)\dot{\psi}_2(t) \right) x_3(t)}{x_3^2(t)} - \] 
\[ - \dfrac{\dot{x}_3(t)(x_2(t) + l)\psi_2(t)}{x_3^2(t)} - \dot{\psi}_3(t) \vee 0, \]
что равносильно
\[\dfrac{\left( \left(-g + \dfrac{x_2(t) + l}{x_3(t)}u^*(t) \right)\psi_2(t) + \left(x_2(t) + l\right)\left(-\psi_1^0 - \dfrac{\psi_{2}(t)}{x_{3}(t)}u^*(t) \right) \right) x_3(t)}{x_3^2(t)} + \]
\[ + \dfrac{(x_2(t) + l)\psi_2(t)}{x_3^2(t)}u^*(t) - \psi_2(t) \dfrac{(x_2(t) + l)}{x_3^2(t)} u^*(t) = \dfrac{- g\psi_2(t) -\psi_1^0(x_2(t) + l)}{x_3(t)} \vee 0 \]
и в конечном счёте сводится к неравенству
\hypertarget{p26}{}
\begin{equation} \label{eq26}
- g\psi_2(t) -\psi_1^0(x_2(t) + l) \vee 0,  \ \ \forall t \in [0, T]. 
\end{equation}
Определим вторую производную $a(t)$
\[ \left( \dfrac{- g\psi_2(t) -\psi_1^0(x_2(t) + l) }{x_3(t)} \right)' =  \dfrac{-g\dot{\psi}_2(t)x_3(t) -\psi_1^0 \dot{x}_2(t)x_3(t) 
+ \dot{x}_3(t) (g \psi_2(t) + \psi_1^0(x_2(t) + l)) }{x_3^2(t)} = \]
\[ = \dfrac{-g \left(-\psi_1^0 - \dfrac{\psi_2(t)u^*(t)}{x_3(t)} \right)x_3(t) -\psi_1^0 \left( -g + \dfrac{(x_2(t) + l)u^*(t)}{x_3(t)} \right)x_3(t)}{x_3^2(t)} - \]
\hypertarget{p27}{}
\begin{equation} \label{eq27}
 - \dfrac{u^*(t)(g\psi_2(t) + \psi_1^0 (x_2(t) + l))}{x_3^2(t)} = \dfrac{-2\psi_1^0 \dot{x}_2(t)}{x_3(t)}. 
\end{equation}

Рассмотрим условие трансверсальности на правом конце, для этого определим конечное множество системы: $x_1(T) = H$, $x_2(T) \in \mathbb{R}$, $x_3(T) \in [M, m_0]$, то есть $\mathcal{X}^1$~--- это полоса в трёхмерном пространстве. Далее разберём два возможных варианта:
\begin{itemize}
	\item В момент времени $T$ мы попали в граничную точку конечного множества. \\
	Очевидно, что это может быть лишь точка 'внизу'  полосы ($x_3(T) = M$), так как, если $x_3(T) = m_0$, значит ракета всё ещё находится на земле. Для данного случая имеем: $\psi_1(T) \in \mathbb{R}$, $\psi_2(T) = 0$, $\psi_3(T) \geqslant 0$.
	\item В момент времени $T$ попали во внутреннюю точку конечного множества. \\
	Тогда $\psi_1(T) \in \mathbb{R}$, $\psi_2(T) = 0$, $\psi_3(T) = 0$.
\end{itemize}

Проведём рассуждения общего плана относительно $\psi_1$. Для этого вспомним выражение для производной $\psi_2(t)$ из сопряженной системы:
\[ \dot{\psi}_2(t) = - \psi_1^0 - \dfrac{\psi_{2}(t)u(t)}{x_{3}(t)}. \]
Если $\psi_1(T) > 0$, то пока $\psi_2(t) > -\psi_1^0 \frac{M}{u_{max}}$, её производная будет оставаться отрицательной. Учитывая, что $\psi_2(T) = 0$, можем считать, что на рассматриваемом временном промежутке $\psi_2(t) \geqslant 0$ и $\dot{\psi}_2(t) < 0$. Аналогично, если $\psi_1(T) < 0$, то $\psi_2(t) \leqslant 0$ и $\dot{\psi}_2(t) > 0$ для $t \in [0, T]$.

Отталкиваясь от условий на $\psi(T)$ определим подозрительные на оптимальность, качественно различные траектории: 
\begin{enumerate}
\item $\psi_1(T) > 0$, $\psi_2(T) = 0$, $\psi_3(T) > 0$. \\
Из сказанного выше $\psi_2(t) \geqslant 0$, а значит, пока $x_2(t) + l \geqslant 0$ из формулы \eqref{eq26} получаем, что $\dot{a}(t) < 0$, то есть управление не возрастает. В конечный момент времени $a(T) =  \hm -\psi_3(T) < 0$, выходит $u^*(T) = 0$. Теперь обратим внимание на производную $\psi_3(t)$:
\[ \dot{\psi}_3(t) = \psi_2(t)u(t) \dfrac{(x_2(t) + l)}{x_3^2(t)}. \]
В силу наших замечаний она имеет такой же знак как и у $x_2(t) + l$, следовательно, $\psi_3(t)$ достигает своего максимального значения при $x_2(t) + l = 0$, причём из условия $\psi_3(T) > 0$ это значение также больше нуля. Тогда в момент времени $\tau$: $x_2(\tau) + l = 0$ имеем: $a(\tau) < 0$, $u^*(\tau) = 0$. Отметим, что при $x_2(t) + l < 0$ у нас $\psi_3(t) > 0$, то есть $a(t) < 0$ и сохраняется нулевое управление. Собрав всё в кучу, мы получим, что при данных граничных условиях на $\psi(t)$ управление не возрастает (строго убывает при значениях не равных 0 или $u_{max}$), при этом имеет ненулевой промежуток времени в конце, на котором обнуляется.

\item $\psi_1(T) > 0$, $\psi_2(T) = 0$, $\psi_3(T) = 0$. \\
Аналогично предыдущему пункту получаем, что в данной ситуации управление не возрастает до 0, но возможно отсутствие нулевого промежутка в конце пути.

\item $\psi_1(T) < 0$, $\psi_2(T) = 0$, $\psi_3(T) > 0$. \\
Из вывода выше $\psi_2(t) \leqslant 0$, а значит, пока $x_2(t) + l \geqslant 0$ из формулы \eqref{eq26} получаем, что $\dot{a}(t) \geqslant 0$, то есть управление не убывает. В конце $a(T) = -\psi_3(T) < 0$, что позволяет нам сделать вывод о существовании продолжительного нулевого управления в конце пути и о том, что обязательно наступит момент, когда $x_2(t) + l < 0$. Итак, здесь управление сначала не убывает до некоторого значения, а после не возрастает до 0 (если внимательно взглянуть на \eqref{eq27}), где пробывает некоторый период времени.

\item $\psi_1(T) < 0$, $\psi_2(T) = 0$, $\psi_3(T) = 0$. \\
Из вывода выше $\psi_2(t) \leqslant 0$, а значит, пока $x_2(t) + l \geqslant 0$ из формулы \eqref{eq26} получаем, что $\dot{a}(t) \geqslant 0$, то есть управление не убывает. В конечный момент времени $a(T) = 0$, выходит $u^*(T) = 0$ и следовательно, обязательно наступит момент, начиная с которого $x_2(t) + l < 0$. По формуле \eqref{eq27} производная $\dot{a}(t)$ сначала возрастает, а после разворота ракеты начинает убывать, заметим также, что $\dot{a}(T) = -\psi_1^0 \dfrac{ (x_2(T) + l)}{x_3(T)} < 0$, это говорит о наличии временного участка $[\tau, T]$, на котором $\dot{a}(t) < 0$. В результате приходим к следующему: управление сначала не убывает до некоторого значения, а после убывает до 0.

\item $\psi_1(T) = 0$, $\psi_2(T) = 0$, $\psi_3(T) > 0$. \\
Из сопряженной системы:
\[ \dot{\psi}_2(t) = - \psi_1^0 - \dfrac{\psi_{2}(t)u(t)}{x_{3}(t)}. \]
Поскольку $\psi_1^0 = 0$, $\psi_2(T) = 0$, то $\psi_2(t) \equiv 0$ на всем временном промежутке. Также из сопряженной системы мы знаем, что
\[ \dot{\psi}_3(t) = \psi_2(t)u(t) \dfrac{(x_2(t) + l)}{x_3^2(t)} \]. 
Откуда получаем, что $\psi_3(t) \equiv \texttt{const} > 0$, тогда $a(t) = -\psi_3(T) < 0$, то есть ракета не взлетит.

\item $\psi_1(T) = 0$, $\psi_2(T) = 0$, $\psi_3(T) = 0$. \\
Аналогично предыдущему пункту получаем, что ракета не тронется.
\end{enumerate}

Теперь обсудим как будет осуществляться перебор параметров для нахождения оптимальной траектории. Отметим, что произвольное решение будет зависеть только от $\psi_1^0$, $\psi_2^0$, $\psi_3^0$. Как видно из вышеприведённого анализа граничных условий мы имеем четыре качественно различных типа подозрительных траекторий:
\begin{enumerate}

\item Не возрастаем до 0 с продолжительным нулевым управлением на конце. \\
В данном случае
 будем вести перебор по двум параметрам: начальное управление $u(0) \in [\frac{m_0 g}{l}, u_{max}]$ и время переключения на нулевое управление $\tau_1^{\text{пер}} \in (0, T)$. Благодаря чему, на каждой итерации у нас будет четыре нелинейных уравнения, связывающих наши неизвестные: 
\hypertarget{p28}{}
\begin{equation} \label{eq28}
\begin{cases} 
4 \cdot u(0)^3 = \psi_2^0 \dfrac{l}{m_0} - \psi_3^0, \\
0 = \psi_2(\tau_1^{\text{пер}}) \cdot \dfrac{x_2( \tau_1^{\text{пер}}) + l}{x_3(\tau_1^{\text{пер}})} - \psi_3(\tau_1^{\text{пер}}),  \\ 
 \int\limits_0^{T}{x_2(\tau)} \, \mathrm{d}\tau = H, \\
 x_3(T) = M, \\ 
\end{cases} 
\end{equation}
первые два получены из условия максимума, последние два~--- граничные условия на $x_1$ и $x_3$. Из-за сложности нахождения решения полученной гамильтоновой системы (\eqref{eq10} совместно с \eqref{eq22}) в явном виде будем решать задачу численно, с помощью метода штрафов и генетического алгоритма. В результате, для каждой точки из заданной выше сетки мы получим оптимальную тройку $\psi_1^0$, $\psi_2^0$, $\psi_3^0$. Затем выберем из них ту, что доставляет минимум рассматриваемому функционалу. Отдельно необходимо разобрать случай старта с $u_{max}$. Здесь имеет смысл сделать перебор по времени $\tau_1^{\text{пер}}$  переключения с $u_{max}$ вместе с $\tau_2^{\text{пер}} > \tau_1^{\text{пер}} $ моментом перехода на нулевое управление, нелинейные уравнения составить по тому же принципу, что и выше.
 
\item Убываем до 0. \\
Данный вариант является частным случаем пункта 1 с $\tau_1^{\text{пер}} = T$ и нефиксированным значением $x_3(t)$ на правом конце.
\item Сначала не убываем, а затем не возрастаем до 0 с периодом нулевого управления в конце. \\
Такие траектории будут учтены в переборе из пункта 1.

\item Сначала не убываем, а затем убываем до 0. \\
Траектории из этого пункта включены в перебор пункта 2.

\end{enumerate}

Подробнее про оптимизацию методом штрафов с генетическим алгоритмом. \\
Для фиксированной пары $u(0)$ и $\tau_1^{\text{пер}}$ перед нами ставится следующая задача минимизации:
\[\mathcal{J}(\psi_1^0, \psi_2^0, \psi_3^0) = \int\limits_{0}^{T}{u^4(t)} \, \mathrm{d}t \ \rightarrow \ \inf,\] 
на множестве 
\[ U = \big{\{} (\psi_1^0, \psi_2^0, \psi_3^0) \in \mathbb{R}^3 \ \big{|} \  (\psi_1^0, \psi_2^0, \psi_3^0) \ \text{удовлетворяет} \ \eqref{eq28} \big{\}}. \]
Мы прибегаем к такой проблеме, так как траекторий, только удовлетворяющих заданным ограничениям, может быть несколько. Нам сложно проанализировать область при такого рода ограничениях, поэтому кажется логичным осуществить переход от исходной постановки на множестве $U$ к последовательности задач минимизации оштрафованной функции на всём $\mathbb{R}^3$:
\[I_k (\psi_1^0, \psi_2^0, \psi_3^0) = \mathcal{J}(\psi_1^0, \psi_2^0, \psi_3^0) + A_k \cdot P(\psi_1^0, \psi_2^0, \psi_3^0) \ \rightarrow \ \inf, \ \ \ k = 1, 2, \dots\]
где $P(\psi_1^0, \psi_2^0, \psi_3^0) = \sum\limits_{i = 1}^4 {\left(g_i(\psi_1^0, \psi_2^0, \psi_3^0)\right)^2}$, а $g_i(\psi_1^0, \psi_2^0, \psi_3^0) = 0$~--- записанное по-другому $i$-е уравнение системы \eqref{eq28}, $A_k > 0$. \\
Все задачи полученной последовательности хотелось бы решить методом безусловной, глобальной оптимизации нулевого порядка, так как возможности эффективно вычислять производные функционалов мы не имеем. Одним из таких методов является генетический алгоритм, правда нам всё же придётся ввести ограничения на принадлежность $\psi_i^0$ некоторому отрезку, дабы обеспечить сходимость метода за разумное время. \\
Генетический алгоритм предлагается реализовать самостоятельно, следующим образом:
\begin{itemize}
\item Каждый экземпляр популяции описывать трёхмерным вектором.
\item Функцию масштабирования принять равной текущему функционалу.
\item Оператор отбора сделать конкурсным.
\item Мутации определить через добавление случайного числа (согласно нормальному распределению) ко всем компонентам вектора.
\item Скрещивание взять одноточечным, либо эвристическим.
\item Остановиться, когда поколения станут мало отличаться друг от друга.
\end{itemize}
Либо воспользоваться уже готовой реализацией $\mathrm{genetic \ algorithm \  tool \ \text{в} \ matlab}$. \\
Если минимум $(\psi_1^{0, k}, \psi_2^{0, k}, \psi_3^{0, k})$, найденный для $k$-ой задачи минимизации таков, что выполняется $A_k \cdot P(\psi_1^{0, k}, \psi_2^{0, k}, \psi_3^{0, k}) < \epsilon$, где $\epsilon$ задаётся заранее, то считаем, что найден минимум исходной задачи и переходим к следующей паре параметров $u(0)$ и $\tau_1^{\text{пер}}$. Иначе, в случае, когда условие не сработало на всех итерациях, мы считаем, что задача неразрешима. \\
Пусть $A_k = 2^{k - 1}$, количество $I_k$, $\epsilon$, а также ограничения на $\psi_i^0$ и некоторые другие вероятностные параметры определим эмпирически. Также отметим, что удобно в генетическом алгоритме в начальную популяцию $I_k$ добавлять минимум $I_{k-1}$.

\item \textbf{Анормальный случай} \\
Пусть $\psi_0^0 = 0$. Заметим, что текущий режим~--- это анормальный случай первой задачи. А значит поиск оптимальной траектории в этом режиме определяется соответствующим перебором, указанным в той задаче.

\item \textbf{Итог} \\
Для нахождения оптимальной траектории достаточно произвести тот же перебор параметров, что и
 в анормальном случае первой задачи, а также проитерироваться по сетке из двух переменных: $u(0) \in [\frac{m_0 g}{l}, u_{max})$, $\tau_1^{\text{пер}} \in [0, T]$, строя соответствующие решения при помощи дополнительных условий и оптимизационных методов. Затем выбрать из них то, у которого достигается минимальное значение рассматриваемого функционала. Отметим также, что случай старта с $u_{max}$ мы учитываем через отдельный перебор по времени $\tau_1^{\text{пер}}$  переключения с $u_{max}$ вместе со временем $\tau_2^{\text{пер}} > \tau_1^{\text{пер}} $ перехода на нулевое управление.
\end{itemize}

\newpage
\begin{thebibliography}{99}
\item Понтрягин~Л.~С., Болтянский~В.~Г., Гамкрелидзе~Р.~В., Мищенко~Е.~Ф. Математическая теория оптимальных процессов. М.: Физматлит, 1961.
\item Чистяков~И.~А. Лекции по оптимальному управлению. 2022-2023.


\end{thebibliography}

\end{document}
